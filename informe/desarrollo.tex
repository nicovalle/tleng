\section{Gramática}

En este trabajo se nos presento una gramática ambigúa con el objetivo de implementar un traductor que a partir de una cadena de símbolos genere un archivo específico para graficar esta expresión matmática de forma más aceptable y formal.

La \textit{Gramática} es la siguiente:

 \begin{equation}
    G = \langle \{ E\};\Sigma;P;E \rangle
 \end{equation}

Donde $\Sigma$ es el alfabeto de la gramática que contiene cualquier caracter salvo \textasciicircum, \_, \{, \}, ( y ).

Mientras que P es el conjunto de \textit{Producciones:}

\begin{center}
 E $\rightarrow$ E E 
\\  $|$ E \textasciicircum E
\\  $|$ E \_ E
\\  $|$ E \textasciicircum E \_ E
\\  $|$ E \_ E \textasciicircum E
\\  $|$ E / E
\\  $|$ ( E )
\\  $|$ \{ E \} 
\\  $|$ $l$
\end{center}

Lo que necesitamos hacer es desambiguar las producciones para que la \textit{gramática} quede desambiguada.

\subsection{Desambiguando}

A partir de las condiciones que debe cumplir la gramática vamos a adaptarla para que quede desambiguada.

\begin{itemize}
 \item ``La división es la de menor precedencia``; por lo que vamos a necesitar una producción nueva para la división
 \item ''seguida de la concatencación``; de igual manera vamos a necesitar otra producción (un \textit{No Terminal} nuevo) para la concatencación
  \item ''Ambas son asociativas a izquierda``; esto implica que la recursión en las \textit{producciones} será \textit{a izquierda}.
  \item ''El superíndice y supraíndice son no asociativos``; entonces van a derivar en valores que no puedan asociarse.
\end{itemize}

\newpage
Siguiendo estas condiciones nos queda el siguiente conjunto de producciones:

\begin{center}
 S $\rightarrow$ E
\\ E $\rightarrow$ E / T
\\ E $\rightarrow$ T
\\ T $\rightarrow$ TF
\\ T $\rightarrow$ F
\\ F $\rightarrow$ G\_G
\\ F $\rightarrow$ G\textasciicircum G
\\ F $\rightarrow$ G\textasciicircum G\_G
\\ F $\rightarrow$ G\_G\textasciicircum G
\\ G $\rightarrow$ \{ E\}
\\ G $\rightarrow$ (E)
\\ G $\rightarrow$ $l$
\end{center}

