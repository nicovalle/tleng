\section{Introducción}

\indent \indent El objetivo de este trabajo práctico fue desarrollar un compositor de fórmulas matemáticas que tomara como entrada una versión simplicada del lenguaje utilzado por LATEX y produjera como salida un archivo SVG.\\
\indent \indent En las siguientes secciones se encontrarán detalles sobre el desarrollo de este trabajo práctico, instrucciones de uso y conclusiones.\\

\newpage

\section{Desarrollo}

\indent \indent La confección de este trabajo podría considerse divida en dos etapas. \\
\indent La primer etapa corresponde a las modificaciones que se realizaron sobre la gramática del trabajo, con el fin de obtener una nueva gramática para la cual se pueda implementar un parser.
\indent La segunda etapa del trabajo significó la implementación del lexer, del parser y de la traducción propiamente dicha. Para ello, se utilizo la herramienta \textbf{ply}, que a partir de una ciertas reglas dadas por nosotros genera un parser LALR.\\

\subsection{Gramática}

\indent Se nos presentó una gramática ambigua a partir de la cual comenzar a trabajar con el fin implementar un traductor de cadenas de símbolos del lenguaje generado por dicha gramática a un fórmula matemática expresada en la sintaxis del formato SVG.
\indent La gramática original es la siguiente:

 \begin{equation}
    G = \langle \{ E\};\Sigma;P;E \rangle
 \end{equation}

\indent Donde $\Sigma = $ \{\_ , / , \{, \}, (, ) , \textit{l}, $\hat{}$ \} , \textit{l} es cualquier caracter a excepción de \_ , / , \{, \}, (, ) , , $\hat{}$  y \textbf{P} es el conjunto de producciones:

\begin{center}
 E $\rightarrow$ E E 
\\  $|$ E \textasciicircum E
\\  $|$ E \_ E
\\  $|$ E \textasciicircum E \_ E
\\  $|$ E \_ E \textasciicircum E
\\  $|$ E / E
\\  $|$ ( E )
\\  $|$ \{ E \} 
\\  $|$ $l$
\end{center}

\indent Al ser ambigua, debimos operar sobre la gramática para desambiguarla, teniendo en cuenta las siguientes restricciones:

\begin{itemize}
 \item La división es la de menor precedencia seguida de la concatencación.
  \item Tanto la división como la concatenación son asociativas a izquierda. Esto implica que la recursión las \textit{producciones} correspondientes será \textit{a izquierda}.
  \item El superíndice y supraíndice son no asociativos. Lo que implica que van a derivar en valores que no puedan asociarse.
\end{itemize}

\indent Con estas cuestiones en mente, a partir de la gramática original se dio lugar a una nueva, no ambigua y que cumple las restricciones solicitadas:\\

\begin{equation}
    G = \langle \{ S, E, T, F, G\};\Sigma;P;S \rangle
 \end{equation}

\begin{center}
 S $\rightarrow$ E
\\ E $\rightarrow$ E / T
\\ E $\rightarrow$ T
\\ T $\rightarrow$ TF
\\ T $\rightarrow$ F
\\ F $\rightarrow$ G\_G
\\ F $\rightarrow$ G\textasciicircum G
\\ F $\rightarrow$ G\textasciicircum G\_G
\\ F $\rightarrow$ G\_G\textasciicircum G
\\ G $\rightarrow$ \{ E\}
\\ G $\rightarrow$ (E)
\\ G $\rightarrow$ $l$
\end{center}

\indent Notar que la producción  S $\rightarrow$ E en realidad podría no estar. Se agregó a efectos de ser coherentes con la implementación que se hizo.\\ 

\subsection{Implementación}

\subsection{Lexer}

\indent \indent El lexer recibe una cadena y devuelve, en caso de poder traducir correctamente, la cadena tokenizada. Los tokens re reconocen haciendo uso de expresiones regulares. Se definieron los siguientes tokens:\\
\begin{itemize}
\item \textbf{SYMBOL}, que representa a cualquier caracter a excepción de \_ , / , \{, \}, (, ) y  $\hat{}$, es decir, representa el \textit{l} de la gramática. Se reconoce con la expresión regular \begin{verbatim} \end{verbatim}.
\item \textbf{LPARENT}, que representa el símbolo (. Se reconoce con la expresión regular \begin{verbatim} [^\_\^\{\}\(\)\/]\end{verbatim}
\item \textbf{RPARENT}, que representa el símbolo ).  Se reconoce con la expresión regular \begin{verbatim} \( \end{verbatim}
\item \textbf{LBRACKET}, que representa el símbolo \{.  Se reconoce con la expresión regular \begin{verbatim} \)\end{verbatim}
\item \textbf{RBRACKET}, que representa el símbolo \}.  Se reconoce con la expresión regular \begin{verbatim} \{\end{verbatim}
\item \textbf{DIVIDE}, que representa el símbolo /.  Se reconoce con la expresión regular \begin{verbatim} \}\end{verbatim}
\item \textbf{CIRCUMFLEX}, que representa el símbolo $\hat{}$.  Se reconoce con la expresión regular \begin{verbatim} \^ \end{verbatim}
\item \textbf{UNDERSCORE}, que representa el símbolo \_.  Se reconoce con la expresión regular \begin{verbatim} \_ \end{verbatim}
\end{itemize}

\indent A continuación se provee el código del lexer. Se puede hallar en el archivo \textbf{lexer\_rules.py}:\\

\begin{verbatim}
tokens = [
	'SYMBOL',
	'LPARENT',
	'RPARENT',
	'LBRACKET',
	'RBRACKET',
	'DIVIDE',
	'CIRCUMFLEX',
	'UNDERSCORE'
]

t_LPARENT = r"\("
t_RPARENT = r"\)"
t_LBRACKET = r"\{"
t_RBRACKET = r"\}"
t_DIVIDE = r"\/"
t_CIRCUMFLEX = r"\^"
t_UNDERSCORE = r"\_"

def t_SYMBOL(token):
	r"[^\_\^\{\}\(\)\/]"
	return token

def t_NEWLINE(token):
    r"\n+"
    token.lexer.lineno += len(token.value)

t_ignore = " \t\n"

def t_error(token):
    message = "Token desconocido:"
    message += "\ntype:" + token.type
    message += "\nvalue:" + str(token.value)
    message += "\nline:" + str(token.lineno)
    message += "\nposition:" + str(token.lexpos)
    raise Exception(message)
\end{verbatim}

\subsection{Parser}

\indent \indent El parser recibe una cadena tokenizada y a devuele un Abstract Syntax Tree de la cadena. Para cada producción de la gramática se define una función. Se crearon distintas clases que representan a los distintos posibles nodos del AST. Su implementación puede observarse en la sección correspondiente a la implementación de la traducción. El código del parser es el siguiente y se halla en el archivo \textbf{parser\_rules.py}:\\

\begin{verbatim}
from lexer_rules import tokens 
from expressions import Start, Divide, Concat, Underscore, Circumflex, 
                                CircumflexUnder, UnderCircumflex, Parenthesis, Symbol

def p_start_expression(subexpressions):
        'start : expression'
        subexpressions[0] = Start(subexpressions[1])
        
def p_expression_divide(subexpressions):
        'expression : expression DIVIDE term'
        subexpressions[0] = Divide(subexpressions[1], subexpressions[3])

def p_expression_term(subexpressions):
        'expression : term'
        subexpressions[0] = subexpressions[1]

def p_term_concat(subexpressions):
        'term : term factor'
        subexpressions[0] = Concat(subexpressions[1], subexpressions[2])

def p_term_factor(subexpressions):
        'term : factor'
        subexpressions[0] = subexpressions[1]

def p_factor_g(subexpressions):
        'factor : g'
        subexpressions[0] = subexpressions[1]

def p_factor_g_under_g(subexpressions):
        'factor : g UNDERSCORE g'
        subexpressions[0] = Underscore(subexpressions[1], subexpressions[3])

def p_factor_g_circum_g(subexpressions):
        'factor : g CIRCUMFLEX g'
        subexpressions[0] = Circumflex(subexpressions[1], subexpressions[3])

def p_factor_g_circum_g_under_g(subexpressions):
        'factor : g CIRCUMFLEX g UNDERSCORE g'
        subexpressions[0] = 
                     CircumflexUnder(subexpressions[1], subexpressions[3], subexpressions[5])

def p_factor_g_under_g_circum_g(subexpressions):
        'factor : g UNDERSCORE g CIRCUMFLEX g'
        subexpressions[0] = 
                     UnderCircumflex(subexpressions[1], subexpressions[3], subexpressions[5])

def p_g_bracket_expression(subexpressions):
        'g : LBRACKET expression RBRACKET'
        subexpressions[0] = subexpressions[2]

def p_g_parenthesis_expression(subexpressions):
        'g : LPARENT expression RPARENT'
        subexpressions[0] = Parenthesis(subexpressions[2])        

def p_g_symbol(subexpressions):
        'g : SYMBOL'
        subexpressions[0] = Symbol(subexpressions[1])

def p_error(subexpressions):
    raise Exception("Syntax error.")
\end{verbatim}

\subsection{Traductor}

\indent \indent  La operatoria para traducir se compone de dos partes. La primera, \textbf{operate} se encarga de calcular a partir del AST, los valores de los atributos para cada nodo. La segunda parte, \textbf{translate}, se corresponde a la traducción propiamente dicha. Esta parte también opera con el AST, al que asume que ya se le aplicó \textbf{operate}, y en base a los valores de los atributos calculados en dicha operación genera la traducción en el formato SVG.\\
\indent Como se podrá ver a continuación, se crearon distintas clases que representan a los posibles nodos del AST. Cada una de ellas posee los siguiente atributos:

\begin{itemize}
\item  \text{x} e \textbf{y} que describen las coordenadas de los nodos y son atributos heredados.
\item \text{width} y \text{height} que representan el ancho y el alto de la subfórmula de cada nodo y son atributos sintetizados.
\item \text{scale} que es un atributo heredado que señala la escala de la subfórmula del nodo.
\end{itemize}

\indent A continuación se provee el código correspondiente a esta sección. Corresponde al archivo \textbf{expressions.py}.
\begin{verbatim}
def scale(node, scale):
    node.scale = scale
    node.height = node.height * scale
    node.width = node.width * scale

def moveY(node, y):
    node.y += y
    for c in node.children:
        moveY(c, y)

def getMaxY(node):
    result = node.y
    for c in node.children:
        childrenY = getMaxY(c)
        result = max(childrenY, result) 

    return result

def getMinY(node):
    result = node.y
    for c in node.children:
        childrenY = getMinY(c)
        result = min(childrenY, result) 

    return result

class Start(object):
    def __init__(self, child):
        self.child = child
        self.x = 0
        self.y = 1
        self.scale = 1
        self.children = [child]

    def name(self):
        return self.child.name()

    def operate(self):
        scale(self.child, self.scale)
        self.child.x = self.x
        self.child.y = self.y
        self.child.operate()

    def translate(self):
        self.translation =  '<?xml version="1.0" standalone="no"?>\n'
        self.translation += '<!DOCTYPE svg PUBLIC "-//W3C//DTD SVG 1.1//EN"\n'
        self.translation += '"http://www.w3.org/Graphics/SVG/1.1/DTD/svg11.dtd">\n'
        self.translation += '<svg xmlns="http://www.w3.org/2000/svg" version="1.1">\n'
        self.translation += '<g transform="translate(0,200) scale(200)" font-family="Courier">\n'
        self.translation += self.child.translate()
        self.translation +='</g>\n'
        self.translation += '</svg>'
        return self.translation

class Divide(object):
    def __init__(self, left, right):
        self.left = left
        self.right = right
        self.width  = 0.6
        self.height = 1
        self.x = 0
        self.y = 0
        self.scale = 1
        self.children = [left, right]

    def name(self):

        return self.left.name() + "/" + self.right.name()

    def operate(self):
        scale(self.left, self.scale)
        scale(self.right, self.scale)
        self.left.x = self.x
        self.left.y = self.y - 0.40 * self.scale
        self.right.x = self.x
        self.right.y = self.y
        self.left.operate()
        self.right.operate()
        rightMinY = getMinY(self.right)
        if (self.y != rightMinY):        
            moveY(self.right, self.right.height - 0.40 * self.scale)
        else:
            moveY(self.right, self.y - 0.40 * self.scale)
        leftMaxY = getMaxY(self.left)
        moveY(self.left, - abs(self.left.y - leftMaxY))
        self.width = max(self.left.width, self.right.width)
        self.height = self.left.height + self.right.height + 2* 0.28 * self.scale
        
        
    def translate(self):
        self.translation = self.left.translate()
        self.translation += 
            '\t<line x1="' + str(self.x) + '" y1="' + str(self.y - 0.28 * self.scale) + '" '
        self.translation += 
           'x2="' + str(self.x + self.width) +'" y2="' + str(self.y -  0.28 * self.scale) + '" '
        self.translation +='stroke-width="0.03" stroke="black"/>\n'
        self.translation += self.right.translate()
        return self.translation



class Concat(object):
    def __init__(self, left, right):
        self.scale = 1
        self.height = 1
        self.left = left
        self.right = right
        self.x = 0
        self.y = 0
        self.width = 0.6
        self.children = [left, right]
        

    def name(self):
        return self.left.name() + " " + self.right.name()

    def operate(self):
        scale(self.left, self.scale)
        scale(self.right, self.scale)
        self.left.x = self.x
        self.left.y = self.y
        self.left.operate()
        self.right.x = self.left.x + self.left.width
        self.right.y = self.y
        self.right.operate()
        self.width = self.left.width + self.right.width
        self.height = max(self.left.height, self.right.height)
        

    def translate(self):
        self.translation = self.left.translate() + self.right.translate()
        return self.translation

class Underscore(object):
    def __init__(self, left, right):
        self.x = 0
        self.y = 0
        self.left = left
        self.right = right
        self.height = 1
        self.scale = 1
        self.width = 0.6
        self.children = [left, right]
        

    def name(self):
        return self.left.name() + "_" + self.right.name()

    def operate(self):
        scale(self.left, self.scale)
        scale(self.right,self.scale * 0.7)
        self.left.x = self.x
        self.left.y = self.y
        self.left.operate()
        self.right.x = self.x + self.left.width
        self.right.y = self.y + (0.25 * self.scale)
        self.right.operate()
        self.width = self.left.width + self.right.width
        self.height = self.left.height + self.right.height - (0.25 * self.scale)

    def translate(self):
        self.translation = self.left.translate()
        self.translation += self.right.translate()
        return self.translation

class Circumflex(object):
    def __init__(self, left, right):
        self.left = left
        self.right = right
        self.scale = 1
        self.height = 1
        self.x = 0
        self.y = 0
        self.width = 0.6
        self.children = [left, right]

    def name(self):
        return self.left.name() + "^" + self.right.name()

    def operate(self):
        scale(self.left, self.scale)
        scale(self.right, self.scale * 0.7)
        self.left.x = self.x
        self.left.y = self.y
        self.left.operate()
        self.right.x = self.x + self.left.width
        self.right.y = self.y - (0.45 * self.scale)
        self.right.operate()
        self.width = self.left.width + self.right.width
        self.height = 
             self.left.height + self.right.height - (0.45 * self.scale)
    
    def translate(self):
        self.translation = self.left.translate()
        self.translation += self.right.translate()
        return self.translation

class CircumflexUnder(object):
    def __init__(self, first, second, third):
        self.scale = 1
        self.x = 0
        self.y = 0
        self.width = 0.6
        self.height = 1
        self.first = first
        self.second = second
        self.third = third
        self.children = [first, second, third]

    def name(self):
        return self.first.name() + "^" + self.second.name() + "_" + self.third.name()

    def operate(self):
        scale(self.first, self.scale)
        scale(self.second, self.scale * 0.7)
        scale(self.third, self.scale * 0.7)
        self.first.x = self.x
        self.first.y = self.y
        self.first.operate()
        self.second.x = self.x + self.first.width
        self.second.y = self.y - (0.45 * self.scale)
        self.third.x = self.x + self.first.width
        self.third.y = + self.y + (0.25 * self.scale)
        self.second.operate()
        self.third.operate()
        self.width = self.first.width + max(self.second.width, self.third.width)
        self.height = 
                self.first.height + self.second.height + 
                self.third.height - (0.45 * self.scale) -  (0.25 * self.scale)
    
    def translate(self):
        self.translation = self.first.translate()
        self.translation += self.second.translate()
        self.translation += self.third.translate()
        return self.translation


class UnderCircumflex(object):
    def __init__(self, first, second, third):
        self.scale = 1
        self.x = 0
        self.y = 0
        self.width = 0.6
        self.height = 1
        self.first = first
        self.second = second
        self.third = third
        self.children = [first, second, third]

    def name(self):
        return self.first.name() + "_" + self.second.name() + "^" + self.third.name()

    def operate(self):
        scale(self.first, self.scale)
        scale(self.second, self.scale * 0.7)
        scale(self.third, self.scale * 0.7)
        self.first.x = self.x
        self.first.y = self.y
        self.first.operate()
        self.third.x = self.x + self.first.width
        self.third.y = self.y - (0.45 * self.scale)
        self.second.x = self.x + self.first.width
        self.second.y = + self.y + (0.25 * self.scale)
        self.second.operate()
        self.third.operate()
        self.width = self.first.width + max(self.second.width, self.third.width)
        self.height = 
               self.first.height + self.second.height + 
               self.third.height - (0.45 * self.scale) -  (0.25 * self.scale)

    def translate(self):
        self.translation = self.first.translate()
        self.translation += self.second.translate()
        self.translation += self.third.translate()
        return self.translation        

class Parenthesis(object):
    def __init__(self, child):
        self.scale = 1
        self.x = 0
        self.y = 0
        self.width = 0.6
        self.height = 1
        self.child = child
        self.children = [child]
        
    def name(self):
        return "(" + self.child.name() + ")"

    def operate(self):
        scale(self.child, self.scale)
        self.child.x = self.x + self.scale * 0.6
        self.child.y = self.y
        self.child.operate()
        self.width = self.child.width + (2 * self.scale * 0.6)
        self.height = self.child.height

    def translate(self):
        self.translation = 
            '\t<text x="0" y="0" font-size="' + str(self.scale) + '" transform="'
        self.translation += 
            'translate(' + str(self.x) + ',' + str(getMaxY(self.child)) +
                  ') scale(1,' + str(self.height) + ')">(</text>\n'
        self.translation += self.child.translate()
        self.translation += 
                 '\t<text x="0" y="0" font-size="' + str(self.scale) + '" transform="'
        self.translation += 'translate(' + str(self.x + self.child.width + self.scale * 0.6) + 
                 ',' + str(getMaxY(self.child)) + ') scale(1,'+ str(self.height) +')">)</text>\n'
        return self.translation

class Symbol(object):
    def __init__(self, value):
        self.value = value
        self.scale = 1
        self.width = 0.6
        self.height = 1
        self.x = 0
        self.y = 0
        self.children = []

    def name(self):
        return str(self.value)

    def operate(self):
        pass

    def translate(self):
        return '\t<text x="' + str(self.x) + '" y="' + str(self.y) +'" font-size="' 
                 + str(self.scale) + '">' + str(self.value) + '</text>\n'
\end{verbatim}


\newpage

\section{Instrucciones de Uso y manual de usuario}

\indent \indent Para correr el programa por línea de comandos ejecutar:\\
\begin{verbatim}
   python traductor.py cadena_a_traducir <archivo_salida>
\end{verbatim}

\indent Este trabajo de desarrolló y se probó con las herramientas:
\begin{itemize}
\item \textbf{Python}, versión 2.7
\item \textbf{Ply}, versión 3.8
\end{itemize}



