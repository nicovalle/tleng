\section{Desarrollo}

\indent \indent La confección de este trabajo podría considerse divida en dos etapas. \\
\indent La primer etapa corresponde a las modificaciones que se realizaron sobre la gramática del trabajo, con el fin de obtener una nueva gramática para la cual se pueda implementar un parser.
\indent La segunda etapa del trabajo significó la implementación del lexer, del parser y de la traducción propiamente dicha. Para ello, se utilizo la herramienta \textbf{ply}, que a partir de una ciertas reglas dadas por nosotros genera un parser LALR.\\

\subsection{Gramática}

\indent Se nos presentó una gramática ambigua a partir de la cual comenzar a trabajar con el fin implementar un traductor de cadenas de símbolos del lenguaje generado por dicha gramática a un fórmula matemática expresada en la sintaxis del formato SVG.
\indent La gramática original es la siguiente:

 \begin{equation}
    G = \langle \{ E\};\Sigma;P;E \rangle
 \end{equation}

\indent Donde $\Sigma = $ \{\_ , / , \{, \}, (, ) , \textit{l}, $\hat{}$ \}  y \textbf{P} es el conjunto de producciones:

\begin{center}
 E $\rightarrow$ E E 
\\  $|$ E \textasciicircum E
\\  $|$ E \_ E
\\  $|$ E \textasciicircum E \_ E
\\  $|$ E \_ E \textasciicircum E
\\  $|$ E / E
\\  $|$ ( E )
\\  $|$ \{ E \} 
\\  $|$ $l$
\end{center}

\indent Al ser ambigua, debimos operar sobre la gramática para desambiguarla, teniendo en cuenta las siguientes restricciones:

\begin{itemize}
 \item La división es la de menor precedencia seguida de la concatencación.
  \item Tanto la división como la concatenación son asociativas a izquierda. Esto implica que la recursión las \textit{producciones} correspondientes será \textit{a izquierda}.
  \item El superíndice y supraíndice son no asociativos. Lo que implica que van a derivar en valores que no puedan asociarse.
\end{itemize}

\indent Con estas cuestiones en mente, a partir de la gramática original se dio lugar a una nueva, no ambigua y que cumple las restricciones solicitadas:\\

\begin{equation}
    G = \langle \{ S, E, T, F, G\};\Sigma;P;S \rangle
 \end{equation}

\begin{center}
 S $\rightarrow$ E
\\ E $\rightarrow$ E / T
\\ E $\rightarrow$ T
\\ T $\rightarrow$ TF
\\ T $\rightarrow$ F
\\ F $\rightarrow$ G\_G
\\ F $\rightarrow$ G\textasciicircum G
\\ F $\rightarrow$ G\textasciicircum G\_G
\\ F $\rightarrow$ G\_G\textasciicircum G
\\ G $\rightarrow$ \{ E\}
\\ G $\rightarrow$ (E)
\\ G $\rightarrow$ $l$
\end{center}

\indent Notar que la producción  S $\rightarrow$ E en realidad podría no estar. Se agregó a efectos de ser coherentes con la implementación que se hizo.\\ 

\subsection{Implementación}

\subsection

